\documentclass[12pt,final]{article}

%% BASIC MATH PACKAGES

\usepackage[utf8]{inputenc}
\usepackage[english]{babel}
\usepackage{cite}
\usepackage{amsthm,amsmath,amssymb,hyperref,amsfonts,adjustbox}
\numberwithin{equation}{section}
\usepackage[shortlabels]{enumitem}
\usepackage[a4paper, total={6in, 8in}]{geometry}
\usepackage{imakeidx}


%%%%%%%%%%%%%%%%%%%%%%%%%%%%%%%%%%%%%%%%%%%%%%%%%%%%%%%
% TIKZ CONFIGURATION FOR THE USE OF QUIVER

\usepackage{tikz}
\usetikzlibrary{cd}
\usetikzlibrary{calc} % `calc` is necessary to draw curved arrows.
\usetikzlibrary{decorations.pathmorphing} % `pathmorphing` is necessary to draw squiggly arrows.

% A TikZ style for curved arrows of a fixed height, due to AndréC.
\tikzset{curve/.style={settings={#1},to path={(\tikztostart)
    .. controls ($(\tikztostart)!\pv{pos}!(\tikztotarget)!\pv{height}!270:(\tikztotarget)$)
    and ($(\tikztostart)!1-\pv{pos}!(\tikztotarget)!\pv{height}!270:(\tikztotarget)$)
    .. (\tikztotarget)\tikztonodes}},
    settings/.code={\tikzset{quiver/.cd,#1}
        \def\pv##1{\pgfkeysvalueof{/tikz/quiver/##1}}},
    quiver/.cd,pos/.initial=0.35,height/.initial=0}

% TikZ arrowhead/tail styles.
\tikzset{tail reversed/.code={\pgfsetarrowsstart{tikzcd to}}}
\tikzset{2tail/.code={\pgfsetarrowsstart{Implies[reversed]}}}
\tikzset{2tail reversed/.code={\pgfsetarrowsstart{Implies}}}
% TikZ arrow styles.
\tikzset{no body/.style={/tikz/dash pattern=on 0 off 1mm}}

%%%%%%%%%%%%%%%%%%%%%%%%%%%%%%%%%%%%%%%%%%%%%%%%%%%%%%%%%%
% Theorem style configuration

\newtheorem{theorem}{Theorem}
\newtheorem{definition}[theorem]{Definition}
\newtheorem{example}[theorem]{Example}
\newtheorem{corollary}[theorem]{Corollary}
\newtheorem{lemma}[theorem]{Lemma}
\newtheorem{proposition}[theorem]{Proposition}

\theoremstyle{remark}
\newtheorem{remark}[theorem]{Remark}

\newcommand*{\im}{\operatorname{im}}
\newcommand*{\dom}{\operatorname{dom}}


%opening
\title{Basic template with a fast guide for vscode latex}
\author{Emmanuel Jerez}


\begin{document}
 \maketitle

 \section{Introduction}

This is my template for my latex writing workflow. Here I'll explain some basic commands for the work with latex in VS Code + latex workshop.

 \section{Theorem, proof and other environments}

 \begin{definition}
  This is a definition
 \end{definition}

 \begin{theorem}
   This is a theorem
 \end{theorem}

 \begin{proof}
  This is the \textbf{proof} environment.
 \end{proof}

 \begin{proposition}
  A proposition
 \end{proposition}

 \begin{corollary}
  A corollary
 \end{corollary}

 \begin{example}
  An example
 \end{example}
 
\begin{remark}
  remark
\end{remark}

\subsection*{Citations}

LaTeX workshop have a citation helper. Inside the command \textbf{cite\{ \}} use the key combination \textbf{Ctrl + Shift + P} and write \textit{citation browser} to open the citation browser. 

This is a cite \cite[Theorem 1]{SOLOMON1967603}.

\subsection*{Jump to PDF and Jump from PDF}

\begin{enumerate}[(J1)]
  \item From the code we use \textbf{Ctrl + Alt + J} to jump to the corresponding PDF part.
  \item From the PDF we do click while holding \textbf{Ctrl} to go from the PDF to the corresponding code part.
\end{enumerate}

 \section{Commutative diagrams with quiver}

 For easy commutative diagram we use quiver, \href{https://q.uiver.app/}{https://q.uiver.app/}

 \[\begin{tikzcd}
	{\mathcal{A}} & {\mathcal{B}} \\
	{ \mathcal{C}} & {\mathcal{C} \otimes \mathcal{B}} \\
	&& \Omega
	\arrow["\phi", hook, two heads, from=1-1, to=1-2]
	\arrow["\eta"', squiggly, from=1-1, to=2-1]
	\arrow["\shortmid"{marking}, maps to, from=1-2, to=2-2]
	\arrow["\psi"', maps to, two heads, from=2-1, to=2-2]
	\arrow["\alpha", curve={height=12pt}, shorten <=11pt, shorten >=17pt, Rightarrow, maps to, from=2-1, to=3-3]
	\arrow["\beta", curve={height=-12pt}, dashed, tail, two heads, from=1-2, to=3-3]
\end{tikzcd}\]

 \bibliographystyle{unsrt}
 \bibliography{azu}

\end{document}
